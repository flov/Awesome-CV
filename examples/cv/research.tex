%-------------------------------------------------------------------------------
%	SECTION TITLE
%-------------------------------------------------------------------------------
\cvsection{Research}


%-------------------------------------------------------------------------------
%	CONTENT
%-------------------------------------------------------------------------------
\begin{cventries}

  %---------------------------------------------------------
  \cventry
    {Master thesis} % Role
    {Invisible to Machine Perception: Attacking Pose Estimators with Attribution Methods} % Event
    {Zurich, Switzerland} % Location
    {Mai. 2020} % Date(s)
    {
      \begin{cvitems} % Description(s)
        \item {
          Neural networks are currently the most accurate techniques to tackle computer vision problems.
          However, with the increase of accuracy, an increase in complexity has emerged leading to black-
          box systems. This is especially problematic in safety-critical situations. In this thesis, we are
          using various attribution methods in 2D or 3D to understand a human pose estimation model.
          During this process, we develop a new method for the 3D attribution case, called 3D Saliency
          map.
          To test the robustness of this model we identify adversarial examples in 2D and propose a new
          method in the 3D domain for computing adversarial texture for clothing. We show that our
          approach works by rendering a video of simulated meshes with and without the adversarial
          texture and feeding into the human pose estimator. To quantify the results we develop three
          different metrics, measuring various aspects of the attacks.
        }
      \end{cvitems}
    }

%---------------------------------------------------------
  \cventry
    {Big Data and Smart Computing (BigComp), 2016 International Conference} % Role
    {Geometry Representations for Big Geometry Data with Unsupervised Feature Learning} % Event
    {Hong Kong, China } % Location
    {Jan. 2016} % Date(s)
    {
      \begin{cvitems} % Description(s)
        \item {
			In this paper, we present an exploration of analyzing geometries via learning local geometry features. After extracting local geometry patches, we parameterize each patch geometry by a radial basis function based interpolation. We use the resulting coefficients as discrete representations of the patches. These are then fed into feature learning algorithms to extract the dominant components explaining the overall patch database. This simple approach allows us to handle general representations such as point clouds or meshes with noises, outliers, and missing data. We present features learned on several patch databases, highlighting the utility of such an analysis for geometry processing applications.
        }
      \end{cvitems}
    }

%---------------------------------------------------------
  \cventry
    {Bachelor thesis} % Role
    {Structure-aware Surface Reconstruction with Sparse Moving Least Squares} % Event
    {Zurich, Switzerland} % Location
    {Aug. 2015} % Date(s)
    {
      \begin{cvitems} % Description(s)
        \item {
        Reconstructing the surface underlying a given point cloud is a fundamental problem in geometry processing. Moving least squares solves this problem efficiently using local fits. However, locality comes at the expense of losing a global view of the geometry, leading to inferior results when there is missing data or significant amount of noise or outliers. Global methods are more robust, but they are expensive to compute. In this thesis, we will combine global and local methods in an efficient manner by using learned local geometry bases and sparse moving least squares fits.
        }
      \end{cvitems}
    }

%---------------------------------------------------------
\end{cventries}
